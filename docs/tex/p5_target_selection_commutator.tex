\documentclass[11pt]{article}
\usepackage{amsmath, amssymb, geometry}
\usepackage{hyperref}
\geometry{margin=1in}

\title{Program for Testing Target Selection Commutativity (P5)}
\author{Alastair J Hewitt}
\date{December 28, 2025}

\begin{document}
\maketitle

\section*{Goal}
To determine if the cosmological inference derived from galaxy clustering is invariant under the ordering of target selection and fiber assignment. In the Causal Evolution (CE) framework, the act of "picking" a target is an event that updates the state of the dependency network. We define the commutator statistic:
\begin{align}
\hat{C} = \mathcal{E}[\text{Clustering} \circ \text{Selection}] - \mathcal{E}[\text{Selection} \circ \text{Clustering}]
\end{align}
Under $\Lambda$CDM, $\hat{C}$ should be consistent with zero (within shot noise). A non-zero $\hat{C}$ suggests that the "Past" (galaxy distribution) is being shaped by the "Context" (survey ledger).

\section*{0. Pre-registration}
\begin{itemize}
\item Data: DESI (Dark Energy Spectroscopic Instrument) Year 1/Year 3 data releases.
\item Ledger: The \texttt{altmtl} (Alternative Merged Target List) suite.
\item Statistic: Monopole and quadrupole of the two-point correlation function $\xi(s)$.
\item Scale: $20 < s < 150 \, h^{-1}\text{Mpc}$ (Linear to quasi-linear regime).
\end{itemize}

\section*{1. Data Products}
\begin{itemize}
\item \textbf{Primary:} DESI LRG (Luminous Red Galaxy) and ELG (Emission Line Galaxy) catalogs.
\item \textbf{Context Ledger:} 128+ realizations of the DESI \texttt{altmtl} fiber assignment pipeline.
\item \textbf{Simulations:} AbacusSummit N-body simulations processed through the fiber-assignment emulator.
\end{itemize}

\section*{2. The Commutator Metric}
We define the test as a comparison between the baseline survey and the ensemble of alternative ledgers.
\begin{enumerate}
\item Let $\theta_{base}$ be the cosmological parameters ($\Omega_m, f\sigma_8$) derived from the primary survey.
\item Let $\theta_{alt, i}$ be the parameters derived from the $i$-th \texttt{altmtl} realization.
\item The commutator shift is defined as:
\begin{align}
\Delta \theta = \theta_{base} - \frac{1}{N} \sum_{i=1}^{N} \theta_{alt, i}
\end{align}
\end{enumerate}

\section*{3. Fiber Assignment Context}
The "Context" in P5 is the physical limitation of the fiber positioners on the focal plane. 
\begin{itemize}
\item \textbf{Fiber Collisions:} Two galaxies closer than the fiber diameter cannot both be observed in one pass.
\item \textbf{Priority Logic:} The ledger decides which galaxy gets the fiber based on a pre-defined "cost" function.
\item \textbf{PbC Hypothesis:} If $\Delta \theta \neq 0$, the priority logic is effectively "writing" the clustering signal rather than just sampling it.
\end{itemize}



\section*{4. Methodology}
\begin{enumerate}
\item \textbf{Clustering Analysis:} Compute $\xi_\ell(s)$ for each \texttt{altmtl} realization using the Landy-Szalay estimator.
\item \textbf{Covariance Estimation:} Use the variance across realizations to define the context-dependent covariance matrix $\mathbf{C}_{ctx}$.
\item \textbf{Likelihood Surface:} Compare the likelihood peaks of the baseline vs. the "average" ledger.
\end{enumerate}

\section*{5. Systematics Triage}
\begin{itemize}
\item \textbf{Angular Selection:} Check if the shift correlates with the survey footprint boundaries.
\item \textbf{Redshift Evolution:} Test if $\hat{C}$ grows at lower redshifts (where the "dependency network" is more complex).
\item \textbf{Shot Noise Calibration:} Use "shuffled" catalogs where fiber positions are randomized (should result in $\hat{C} = 0$).
\end{itemize}

\section*{6. Expected Sensitivity}
DESI's volume is sufficient to detect "bookkeeping coupling" at the level of $\lambda \sim 10^{-3}$ in the growth rate of structure $f\sigma_8$.

\section*{7. Stop-loss Rules}
\begin{itemize}
\item If $\Delta \theta$ is within the $1\sigma$ scatter of the AbacusSummit mocks, the null hypothesis ($\Lambda$CDM) is retained.
\item If the shift is only present in ELGs but not LRGs, investigate species-specific selection bias.
\end{itemize}

\end{document}
