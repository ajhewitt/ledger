\documentclass[11pt, a4paper]{article}
\usepackage[utf8]{inputenc}
\usepackage{amsmath, amssymb, amsfonts}
\usepackage{graphicx}
\usepackage{booktabs}
\usepackage{hyperref}
\usepackage{geometry}
\usepackage{setspace}
% \usepackage{cite}

% Formatting for readability (Single Column, 1.15 spacing)
\geometry{margin=1in}
\setstretch{1.15}

\title{\textbf{Cosmological Decoherence: \\ A Multi-Epoch Audit of Context-Coupling in the Universal Ledger}}
\author{Alastair J Hewitt\\ \small{The Ante Institute}}
\date{December 28, 2025}

\begin{document}

\maketitle

\begin{abstract}
    \noindent Standard cosmological inference relies on the assumption of \textit{Temporal Evolution} (TE), where the universe is modeled as an objective territory independent of the observer. However, persistent large-scale anomalies---such as the ``Axis of Evil'' in the CMB---challenge this assumption. We propose the \textbf{Parochial by Construction (PbC)} framework, rooted in Causal Evolution (CE), which posits that the observed record is a dependency network phase-locked to the observer's context. 
    
    We present results from a three-pillar audit of the cosmic ledger: 1) A differential audit of Planck NPIPE polarization maps reveals a $15.0\sigma$ residual alignment with the scan strategy; 2) A variance audit of COSMOS2020 galaxies yields $83.88\sigma$ significance relative to noise, consistent with standard gravitational instability; and 3) A commutator test of DESI DR1 samples yields a null result ($1.43\sigma$). This progression defines a \textbf{Cosmological Decoherence Curve}. Finally, we propose a falsifiable prediction for the upcoming LiteBIRD mission (2032): that the ``Axis of Evil'' will rotate relative to the Planck frame, tracking the change in scan strategy.
\end{abstract}

\section{Introduction: The Epistemic Crisis}

The standard $\Lambda$CDM model faces epistemic friction on the largest scales. The alignment of the Cosmic Microwave Background (CMB) multipoles with the local Solar System geometry---the ``Axis of Evil''---remains a statistically significant anomaly ($p < 0.05\%$). Standard interpretations treat this as a fluke or an undetected systematic.

We propose an alternative: the ``Past'' is not a pre-existing territory, but a reconstructed resource. In the high-redshift regime, the reconstruction relies heavily on the observational Context ($\hat{T}_{ctx}$), creating a non-zero commutator:
\begin{equation}
    [\hat{R}, \hat{T}_{ctx}] = i \hbar_{pbc} \hat{C}
\end{equation}
This paper tests the hypothesis that $\hat{C}$ is non-zero in the early universe and decays to zero in the local universe.

\section{Theoretical Framework: The Dual Construction}

\subsection{The Woodbury Inversion}
In the regime where noise (context) dominates the signal (territory), the standard Wiener filter inversion is governed by the Woodbury Matrix Identity. This allows us to expand the posterior into a \textbf{Dual Construction}:
\begin{equation}
    \hat{s}_{dual} = S (S + N)^{-1} d
\end{equation}
The estimator is a projection of the Total Covariance $(S+N)$ onto the Signal prior $S$. 

\subsection{The Ledger Term}
We define the mixing component $\mathcal{L} = S N^{-1} S$ as the ``Ledger Term.'' If the eigenbasis of the Signal ($S$) is aligned with the eigenbasis of the Noise ($N$), the reconstruction cost is minimized.
\begin{itemize}
    \item \textbf{Phase-Locking:} When $\mathcal{L}$ dominates, the record $\hat{s}$ mirrors the geometry of $N$ (e.g., the Planck scan strategy).
    \item \textbf{Decoherence:} As the signal variance $S$ grows (via structure formation), the term $(S+N)$ becomes dominated by $S$, and the dependency on $N$ vanishes.
\end{itemize}

\section{Methodology: The Three-Pillar Audit}

\subsection{P1: Primordial Phase-Locking (Planck)}
We audit the Planck NPIPE (PR4) 143 GHz polarization maps. The ``Context'' is defined by the Hits Map $H(\hat{n})$. To isolate physical coupling from additive noise, we use the \textbf{Half-Ring (HR) Difference} method. Since both Half-Rings share the same scan path, scan-synchronous physical signals cancel in the difference map but sum in the signal map.
\begin{equation}
    \Delta S_{\gamma} = \rho(\psi_{sum}, H) - \rho(\psi_{diff}, H)
\end{equation}

\subsection{P6: Structuring Variance (JWST/COSMOS)}
We audit the COSMOS2020 Classic catalog at $z \sim 4.5$ ($N=86,108$). We employ a \textbf{Mask-Aware Grid} ($10 \times 10$) to calculate the Variance Ratio ($V = \sigma^2 / \mu$), filtering for patches with $>90\%$ occupancy to remove edge effects. We compare the observed variance against Poisson Null ($\mathcal{H}_0$) and $\Lambda$CDM Mocks ($\mathcal{H}_{grav}$).

\subsection{P5: Late-Time Stationarity (DESI)}
We audit the DESI DR1 LRG sample ($z \sim 0.7$). We test for stationarity by measuring the redshift shift $\Delta z$ induced by altering the target selection logic.

\section{Empirical Results}

\subsection{P1: The Robust Residual \texorpdfstring{($15.0\sigma$)}{(15.0 sigma)}}
The differential audit reveals a net coupling of $\Delta S_{\gamma} = 15.0\sigma$. While standard analysis attributes this to ``Intensity-to-Polarization'' leakage, the persistence of the signal implies the primordial record is physically phase-locked to the observer's path.

\subsection{P6: Gravity Takes Over \texorpdfstring{($83.88\sigma$)}{(83.88 sigma)}}
The Mask-Aware audit yields $V_{obs} = 15.39$, an $83.88\sigma$ deviation from randomness. However, comparison with clustering mocks ($V_{mock} \approx 28.4$) suggests this signal is consistent with standard gravitational instability. This marks the epoch where the Territory (Gravity) begins to dominate the Context.

\subsection{P5: The Balanced Ledger \texorpdfstring{($1.43\sigma$)}{(1.43 sigma)}}
The DESI audit yields a shift of $\Delta z \approx -4.7 \times 10^{-4}$ ($1.43\sigma$). By $z=0.7$, the ledger is stationary; the observer's bookkeeping no longer biases the physical record.

\section{Discussion: The Decoherence Curve}

The progression of significance defines the history of the universe's dependency on the observer:
\begin{itemize}
    \item \textbf{Primordial ($z \sim 1100$):} $15.0\sigma$ (Context-Dominated)
    \item \textbf{Structuring ($z \sim 4.5$):} $83.9\sigma$ (Transition/Gravity)
    \item \textbf{Local ($z \sim 0.7$):} $1.4\sigma$ (Territory-Dominated)
\end{itemize}

\section{Falsifiable Predictions}

To distinguish the PbC framework from a mere reinterpretation of systematics, we propose a specific falsification test based on the upcoming LiteBIRD mission.

\subsection{The LiteBIRD Rotation Test}
LiteBIRD is scheduled for launch in 2032 via an H3 vehicle to the Sun-Earth Lagrangian point L2. While its orbit is similar to Planck's, its scan strategy differs significantly ($\alpha \approx 50^\circ$ vs Planck $\alpha \approx 85^\circ$).

\textbf{Prediction:} If the ``Axis of Evil'' is a property of the Territory (Standard Model), its orientation on the sky must remain invariant. If it is a Parochial Phase-Lock (PbC), the axis will \textbf{rotate} in the LiteBIRD maps to track the new principal components of the scan covariance. We predict the axis will shift by an angle $\theta_{rot} \neq 0$.

\section{Conclusion}
We have presented a unified audit of the cosmic ledger. By isolating the $15\sigma$ residual in the CMB and contrasting it with the gravity-dominated structure at later times, we validate the PbC framework. The universe is not a static place, but a decohering process that evolves from a subjective record to an objective reality.

\bibliographystyle{plain}
\bibliography{pbc_refs}

\end{document}
